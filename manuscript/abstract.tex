\begin{abstract}

Microbial transfer of both pathogenic and non-pathogenic strains from the environment can influence a person's health, but such studies are rare and the phenomenon is difficult to study. Here, we use the unique, isolated environment of the International Space Station (ISS) to track environmental movement of microbes in an astronaut's body. We identified several microbial taxa, including \textit{Serratia Proteamaculans} and \textit{Rickettsia australis}, which appear to have been transferred from the environment of to the gut and oral microbiomes of the on-board astronaut, and also observed an exchange of genetic elements between the microbial species. Strains were matched at the SNP and haplotype-level, and notably some strains persisted even after the astronaut's return to Earth. Finally, some transferred taxa correspond to secondary strains in the ISS environment, suggesting that this process may be mediated by evolutionary selection, and thus, continual microbial monitoring can be important to future spaceflight mission planning and habitat design.


\end{abstract}

\section{Introduction}

Human commensal microbiomes have a known hereditary component \citep{Goodrich2016}, but the non-hereditary, acquired portion of the human microbiome is mediated by a large number of factors. An ideal study for microbial transfer would utilize a longitudinal sampling of subjects in a hermetically-sealed environment that was already profiled with strain-level resolution. The microbiome can  change as a function of age, developmental stage, environmental exposures, antibiotic use, diet, and lifestyle, yet strain-level mapping and longitudinal tracking of such dynamics are limited. In particular,  the movement of non-pathogenic microbes and how they can colonize an adult commensal microbiomes in a defined, quantified, and hermetically-sealed environment, is almost completely unknown \citep{Schwendner2017}.

Evidence for the colonization of adult commensal microbiomes by environmental microbes could have important health implications, as it would provide a mechanism for how regional environmental microbiomes impact a person's microbiome. Cities in particular are known to host diverse environmental microbiomes \citep{danko2019global} and transfer between commensal and environmental microbiomes may add to explanations for health differences between otherwise similar regions \citep{Nicolaou2005}. The selective transfer of certain microbial strains may also carry evolutionary implications if a microbial species can be shown to follow distinct selective patterns inside and outside of human commensal microbiomes. 

The ISS presents several advantages for the study of microbial transfer. As an environment, the ISS is heavily studied (as are its occupants), it is a uniquely sealed environment with little chance of infiltration by exterior microbes between regular supply missions, and microgravity may lead to a general diffusion of microbial particulates not present in more ordinary environments. Here,  we present evidence for the colonization of an adult's gut and oral microbiome by environmental strains while on the International Space Station (ISS), during an almost year-long mission\citep{garrett2019nasa}. Of note, several of these strains were continuously observed after the mission, providing evidence of a persistent influence on the astronaut's microbiome, which may help to inform future studies.
