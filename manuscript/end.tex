\section{Conclusion}

We have identified genetic evidence of microbial transfer between the fecal and saliva microbiomes of an adult and between these microbiomes and their environment. If confirmed this result demonstrates that non-pathogenic microbes from the environment can colonize adults and establishes the possibility of ongoing microbial flux between humans and their environment. 

A number of open questions remain. We have made a first attempt to quantify the rate of transfer between different microbiomes and given an estimate for the total number of emergent species in a gut microbiome which cannot be explained as the result of repeated sampling alone. However, these estimates necessarily suffer from the small sample sizes available in this study and the unusual situation under which the samples were taken. To conclusively establish the scope of microbial transfer will require broader studies as well as confirmation using culture based techniques. That said the unusual nature of spaceflight provides as strongly controlled an environment as is likely to be possible making this a near optimal model set up to study microbial transfer.

Confirmation of our result and improved estimates for transfer rates will be beneficial to public health. Non-pathogenic microbial exchange with the environment represents a significant unknown for its impact on human health. Accurate quantification of this impact can lead to targeted interventions and may also shed light on issues like city enriched diseases and the hygiene hypothesis.

\section{Acknowledgment}

TODO