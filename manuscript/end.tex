\section{Conclusion}

We have identified genetic evidence of microbial transfer between the fecal and saliva microbiomes of an adult and between these microbiomes and their environment. These results demonstrate that non-pathogenic microbes from the environment can colonize adults and establishes the possibility of ongoing microbial flux between humans and the unique ISS environment. Moreover, these provide candidate "ISS mobile" species and also enable a key estimate of the fraction of taxa that could be transferred from different sources of the body while in the spaceflight environment.

A number of open questions remain. We have made a first attempt to quantify the rate of transfer between different microbiomes and given an estimate for the total number of emergent species in a gut microbiome which cannot be explained as the result of repeated sampling alone. However, these estimates necessarily suffer from the small sample sizes available in this study and the unusual situation under which the samples were taken. To conclusively establish the scope of microbial transfer will require broader studies as well as confirmation using culture-based techniques. Nonetheless, the unusual nature of spaceflight provides as strongly controlled an environment as is likely to be possible making this a near-optimal model set up to study microbial transfer.

The emergence of novel taxa, while intriguing, must be placed into the context of expected stool sampling variation. To account for such sampling dynamics, we also conducted a rigorous re-sampling study. Our data showed that TW and HR had more newly observed taxa at some (but not all) of the time points relative to the 100,000 subset. Importantly, the number of new taxa that were observed in subsets dropped off quickly for later time points as the subsets reached saturation. Subsets generally showed an adversarial selection, wherein many new taxa at one time point would lead to fewer new taxa at later time points. The 243 fecal replicates had similar read counts to the time series from HR and TW, and this should not be an issue, but could also be examined in greater detail in future studies.

Of note, repeated sampling can identify low abundance species which were dropped out of previous samples and because different sample preparation techniques can yield different sets of taxa. A series of samples taken from a microbiome that is exchanging taxa with an external environment will have an additional source of new taxa. These taxa would not be identified in earlier samples because they were not present, and this is another source of variation that could be mapped and quantified for future missions (more sampling of more areas of the body and the ISS, and at greater depth).

Taken together, the matching genomic regions and SNPs within the haplotypes strongly supports the conclusion that novel taxa in pre-flight commensal microbiomes from TW could come from the environment or from other commensal microbiomes. The size of transferred regions and number of SNPs suggests that "taxa transfer" between commensal microbiomes occurs more frequently than they transfer from the environment to commensal microbiomes. However, these rates may prove to be anomalous for either TW, habitation in the ISS, or both, since non-pathogenic microbial exchange with the environment represents a significant unknown for its impact on human and astronaut health. Nevertheless, accurate quantification of microbial strains and their movements can lead to targeted interventions, shed light on the hygiene hypothesis (broadly and on the ISS), and help in planning for future missions and astronaut monitoring.

\section{Availability and Access}

All analysis and figure generating code may be found on GitHub at \url{https://github.com/dcdanko/twins_iss_transfer}. All results and raw data may be found on Pangea at \url{https://pangea.gimmebio.com/sample-groups/62661efb-a433-4ae5-bcec-de704a80e217}.

\section{Acknowledgment}

We would like to thank the Epigenomics Core Facility at Weill Cornell Medicine, the Scientific Computing Unit (SCU), XSEDE Supercomputing Resources, the Starr Cancer Consortium (I13-0052), the Vallee Foundation, the WorldQuant Foundation, The Pershing Square Sohn Cancer Research Alliance, NASA (NNX14AH50G, NNX17AB26G), the National Institutes of Health (R01AI151059), TRISH (NNX16AO69A:0107, NNX16AO69A:0061), the Bill and Melinda Gates Foundation (OPP1151054), 
 and the Alfred P. Sloan Foundation (G-2015-13964).


