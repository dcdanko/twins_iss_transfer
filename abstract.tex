\begin{abstract}

\textbf{Motivation}  There is little direct evidence for the transfer of non-pathogenic microbes from the environment into adult human commensal microbiomes. If it exists and is common such transfers would have important health implications. Different geographic regions contain different environmmental microbes and, via transfer to commensal microbiomes, these microbes could contribute to geographic variation in health and subclinical conditions.

\textbf{Results} We identify several microbial taxa which appear to have been transferred from the environment of the International Space Station (ISS) to the gut and oral microbiomes of an on-board astronaut. We also identify transfer between these microbiomes. We confirm that these taxa are identical strains at the SNP level and show that some of these persist after the astronauts return to earth. Some transferred taxa correspond to secondary strains in the ISS environment suggesting a process mediated by evolutionary selection.

\textbf{Availability} All data used for this study is publicly available. All analysis and figure generating code is available on GitHub.

\end{abstract}

\section{Introduction}

Human commensal microbiomes have a strong hereditary component \citep{Goodrich2016}. It is not clear, however, where the non-hereditary portion of the human microbiome emerges. There are significant changes to commensal microbiomes in early life but it is unknown if or how the adult commensal microbiome can be altered. In particular it is unknown if non-pathogenic microbes can colonize an adult commensal microbiome.

Evidence for the colonization of adult commensal microbiomes by environmental microbes would have public health implications as it would provide a mechanism for how regional environmental microbiomes impact health. Cities in particular are known to have diverse environmental microbiomes \citep{danko2019global} and transfer between commensal and environmental microbiomes may add to explanations for health differences between otherwise similar regions \citep{Nicolaou2005}. The selective transfer of certain microbial strains may also carry evolutionary implications if a microbial species can be shown to follow distinct selective patterns inside and outside of human commensal microbiomes. 

In this paper we present data which we believe gives evidence for the colonization of an adults gut and oral microbiomes by environmental microbiomes. The environment in question is the International Space Station (ISS) and the adult is an astronaut present on the station for nearly one year. The ISS presents several advantages for the study of microbial transfer. As an environment the ISS is uniquely heavily studied (As are its occupants), it is a uniquely sealed environment with little chance of infiltration by exterior microbes between regular supply missions, and micro-gravity may lead to a general diffusion of microbial particulates not present in more ordinary environments. 
